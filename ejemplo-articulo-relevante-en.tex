\documentclass{sistedes}

\begin{document}

\title{Contribution Title\thanks{Supported by organization x.}}

\author{
First Author\inst{1}\orcidID{0000-1111-2222-3333}
\and
Second Author\inst{2,3}\orcidID{1111-2222-3333-4444}
\and
Third Author\inst{3}\orcidID{2222--3333-4444-5555}
}

\institute{
Dummy University, Nowhere, Spain\\
\email{first@example.com}\\
\and
Great University, Nowhere, Spain\\
\email{second@example.com}\\
\url{http://www.example.com/second}
\and
Dummy Institute, Nowhere, Spain\\
\email{\{second,third\}@example.com}
}

\maketitle

\small

\keywords{keyword 1, keyword 2, keyword 3, ...}

%% Specify where this paper was published

% For journal papers, uncomment the command below indicating the journal name, volume, issue, pages and year
\publishedin{Journal name, Vol. X, Issue Y, No. Z, pp. xx--yy, YEAR}

% For conference papers, uncomment the command below indicating the acronym of the conference, year, proceedings name, and pages
%\publishedin{Acronym YEAR - Proceedings of full conference name, pp.  XXX--YYY, YEAR.}

%% Specify impact information of your publication.
%% Below, you can find some templates that may be used to justify the impact of your publication
%% If none of the templates below fits your needs, you may want to include any other indicator
%% that allows verifying the quality of your work according to the CoARA principles (https://coara.eu/coalition/guiding-principles/)

% For journal papers, uncomment the command below indicating the JCR IF, quartile, position, and area
\impact{JCR X.XX - QX - Position: XX/YY  - Area: Area / Subarea}

% For conferences, uncomment the command below indicating the GGS class 
%\impact{GII-GRIN-SCIE Class X (Rating)}

% If your conference does not require to provide impact information, you can just remove the 
% the \impact{...} commands above 


\DOI{https://doi.org/xxxxx/xxxxx}


\begin{abstract}
% Here, the abstract
\lipsum[1-2]
\end{abstract}

\end{document}
